%Het eerste hoofdstuk van je thesis.
\chapter{Literatuurstudie}
Dit hoodstuk verduidelijkt de theorie en de kennsi die nodig is om het onderzoek in de masterproef te berkijpen. Als eerste bespreekt men de werking van satellieten in sectie \ref{LSat}. Vervolgens wordt er een beeld van de satelliet constellaties die gebruikt worden geschetst. Deze worden besproken in \ref{LGNS}. Eens we de verschillende types van satelieten kennen, besrpeken we de manier waarop ze met elkaar communiceren in sectie \ref{LCom}.

\section{Satellieten}
\label{LSat}
Om de werking van de satelliet systemen beter te begrijpen, verklaren we in deze sectie eerst hoe men aan de hand van satellieten de positie kan bepalen in \ref{LPbS}. Vervolgens bepsreken we de verschillende types van satellieten in \ref{LVTS} die voorkomen in de constellaties die besproken worden in \ref{LGNS}. Satellieten worden geobserveerd en gestuurd door grond stations. De zorgen voor de spatiale informatie in het signaal. Dit is de ephemeris data (de baan van een satelliet) of de almanac data (de realtie tussen alle satellieten). Bijkomend wordt ook informatie over de klok verzonden \cite{LBibGNSS8}.

\subsection{Positiebepaling aan de hand van satellieten}
\label{LPbS}
Het algemeen princiepe achter een satelliet navigatie systeem ide de trilateratie vanuit elk punt op het oppervlak van de aarde ten op zichte van de zichtbare satellieten.  De afstand tot de satellieten wordt gemeten door de tijd die het radio signaal nodig heeft om de ontvanger te bereiken. Omdat het radiosignaal reist aan de licht snelheid, worden zeer nauwkeurige klokken gebruikt. De satellieten bevatten atomische klokken, de ontvangers geavanceerde quartz klokken. De exacte locatie van de satelliet is een gekende voor de procedure. Dit is mogelijk omdat de banen van de satellieten zeer stabiel en voorspelbaar zijn. In princiepe zijn drie satellieten voldoende om een dire dimensionale positie te bepalen. Alle punten met de zelfde afstand tot een satelliet voren een sferisch oppervlak met de satelliet in het midden. Drie sferische oppervlakken snijden elkaar in twee punten. E\'en van deze punten kan verwaarloosd worden omdat de positie te ver is van het aardoppervlak. Dit is grafisch weergegeven in figuur \ref{imgPbS} Een vierde signaal is nodig om de tijdsverschillen te elimineren tussen de klokken van de satellieten en de ontvanger. Het aantal ontvangers is niet gelimiteerd omdat gebruikers zich op een passieve manier gedragen \cite{LBibGNSS8}. 
\begin{figure}[hpb]
	\includegraphics[scale=1.75]{BepalingPositie.jpg}
	\caption{Positie bepaling aan de hand van satellieten \cite{LBibSat}}
	\label{imgPbS}
\end{figure} 
De grootste oorzaken van fouten zijn de volgende \cite{LBibGNSS8}:
\begin{itemize}
	\item De geometrische positie van de satellieten
	\item Klok fouten
	\item Ephemerische fouten
	\item Troposferische en ionosferische condities
	\item Multipad effecten
	\item Onjuistheden van de ontvanger
\end{itemize}

\subsection{Verschillende types satellieten}
\label{LVTS}
\subsubsection{In Orbit Validation satellite}
\subsubsection{Geostationary Earth Orbit Satellilte}
\subsubsection{Inclined Geop-Synchronous Orbit satellit}
\subsubsection{Medium Earth Orbit satellite}

\section{GNSS}
\label{LGNS}
GNSS ofwel Global Navigation Satellite System, is een sateliet systeem met een wereldwijde dekking. GNSS is een standaard term gebruikt om systemen te beschrijven die positie en navigatie oplossingen bieden. Het werd voornamelijk ontwikkeld voor de luchtvaart en ruimtevaart industrie en militaire doeleinden. Tegenwoordig worden deze technologie\"en ook in andere toepassingen gebruikt. GNSS vraagt samenwerking tussen verschillende publieke en private organisaties \cite{LBibGNSS3}.  Dit systeem is opgebouwd uit verschillende subsystemen die verder in deze tekst bepsroken worden. Namelijk: GPS besproken in sectie \ref{LGPS}, In sectie \ref{LGLO} bespreken we GLONASS, dit zijn beiden miltitaire systemen. Toch zijn ze beschikbaar voor gebruik door internationale private en commerci\"ele doeleinden \cite{LBibGNSS8}. Galileo wordt besproken in sectie \ref{LGal} en BeidDou komt al laatste aan bod in sectie \ref{LBeD}. Deze subsystemen zijn opgebouwd uit drie delen. Nameijk het ruimte gedeelte, het controle gedeelte en het gebruikers gedeelte \cite{LBibBeiDou2}.  Het IGS, International GNSS Service staat in voor de aflevering van de hoogste kwaltieit GNSS data en producten \cite{LBibGNSS}. Elke satelliet is een project op zichzelf. Hierdoor is het moeilijk om een gestandaardiseerde productie ketting te creëren \cite{LBibGNSS3}. Om de performantie van GNSS te meten zijn beschikbaarheid, betrouwbaarheid en nauwkeurigheid sleutel parameters. Met de beschikbaarheid bedoelt men het aantal satellieten dat zichtbaar is vanaf de positie van de gebruiker. De nauwkeurigheid is een maat voor hoe dicht de navigatie oplossing die door het systeem voorzien wordt is bij de echte locatie en snelheid van de gebruiker \cite{LBibGNSS6}. De nodige nauwkeurigheid is afhankelijk van de toepassing waarin het gebruikt wordt \cite{LBibRTK3}. Gebruikers merkten al in het begin op dat de satellieten van een constellatie in het zelfde gedeelde van de hemel voorkomen met een periode die net iets kleiner is dan \'e\'en dag. Dit was een belangrijke eigenschap toen het systeem nog niet volledig was en data verzameld moest worden op het moment dat de meeste satellieten zichtbaar waren \cite{LBibGNSS7}. Ondertussen is het systeem onmisbar en komen er steeds meer extra toepassingen bij \cite{LBibGNSS8}.

\subsection{GPS}
\label{LGPS} 
GPS staat voor Global Positioning System. Dit is het satelieten gebaseerde positie systeem dat bij ons het bekendste is. Het is ontwikkeld in Amerika \cite{LBibGNSS}\cite{LBibGNSS3}. het is \'e\'en van de langst en meest gebruikte systemen binnen GNSS en draait momenteel op volledige capaciteit \cite{LBibGNSS4,LBibGNSS8}. GPS wordt vaak gebruikt als algemene term voor satelliet navigatie. Het primair doel van GPS is drie dimensionale navigatie \cite{LBibGNSS8}. Voor de gebruikers is kost van de data een hoge bezorgdheid. Daarom is het belangrijk om de kosten van de operaties te verkleinen. GPS moet toegankelijk zijn voor meerdere clients. De gebruikerstoegang gebeurt voornamelijk door mobiele toestellen die verbindig maken via het internet \cite{LBibGPS}. Voor sommige toepassingen voldoen de nauwkeurigheid en de betrouwbaarheid van de GPS constellatie niet \cite{LBibGNSS6}.

\subsubsection{Opbouw constellatie}
 De constellatie van GPS bestaat uit 32 satellieten en heeft zijn volledgie functionele capaciteit bereikt (FOC), het systeem wordt geleidelijk aan gemoderniseerd \cite{LBibGNSS4}. De satellieten zijn geplaatst in een bijna circulaire baan met een straal van 2562.75km. De periode bedraagt 11 uur en 58 minuten \cite{LBibGNSS6,LBibGNSS8}. Dit is de tijd die verstrijkt tussen dat een satelliet terug op dezelfde plaats wordt waargenomen. De satellieten worden geplaatst in 6 orbitale vlakken. De vlakken zijn gelijk geplaatste in de lengte richting. Maar de satellieten in elk vlak zijn niet op gelijke afstand geplaatst \cite{LBibGNSS6}. Deze constellatie voor ziet de gebruiker met vijf tot acht zichtbare satellieten vanop elk punt op het aardoppervlak \cite{LBibGNSS8}. De opbouw van de constellatie wordt grafisch weergegeven in figuur \ref{imgGPS}.
 
 \begin{figure}[hpb]
 	\includegraphics[scale=1]{GPS.png}
 	\caption{GPS constellatie \cite{LImgGPS}}
 	\label{imgGPS}
 \end{figure}

\subsubsection{Signalen}
Het GPS signaal bestaat uit twee zeer stabiele, bijna monochromatische draaggolven L1 en L2, waarop drie modulaties op aanwezig zijn:
\begin{itemize}
	\item C/A code
	\item P code
	\item broadcast bericht
\end{itemize}. 
Alle componenten van een GPS signaal zijn gebaseers op een fundamentele kloksnelheid f\textsubscript{0} van 10,23 MHz. De GPS draaggolven hebben een frequentie van 154 f\textsubscript{0} voor L1 en een frequentie van 120 f\textsubscript{0} voor L2\cite{LBibGPS2}. GPS signalen zijn modulaties van de draaggolven L1 en L2 voor alle satellieten \cite{LBibGPS3}.

\subsubsection{Manieren om posities te bepalen}
\paragraph{Differential GPS Corrections}
Differential GPS Corrections ook weel DGPS genoemd, worden gebruikt om nauwkeurig posities  te bepalen \cite{LBibGLONASS2}. De zelfde techniek wordt ook gebruikt op GNSS niveau, men spreekt dan van DGNSS \cite{LBibGNSS8}. Er worden minimaal twee GPS ontvangers gebruikt. Van \'e\'en van deze ontvangers moet de preciese postie gekend zijn, dit noemen we de basis ontvanger \cite{LBibGNSS2,LBibRTK}. De basis ontvanger gaat data verzenden over een radio link.  Het tweede station kan in bewging zijn, dit is echter geen voorwaarde. Het tweede station berekend zijn positie aan de hand van de data hij ontvangt van de satellieten maar ook van de data die het krijgt via de radio link van het basis station \cite{LBibRTK}. Dit is een populaire manier om sateliet en klok fouten te elimineren. Een nadeel van deze techniek is dat de observaties van beide stations simultaan moeten gebeuren \cite{LBibGNSS2}. Deze techniek geeft meedstal een resultaat op 1m nauwkeurig \cite{LBibRTK}. Deze metingen gebeurden de pseodo range data, ze worden niet real-time gemaakt \cite{LBibRTK3}.

\paragraph{Real-Time Kinematic}
Real-Time Kinematic wordt vaak afgekort als RTK. RTK is een speciale vorm van DGPS dat ongever op centimeter nauwkerigheid werkt. Indien je door gebruik te maken van dit algoritme de positie wilt bepalen, moeten er minimaal vijf satellieten zichtbaar zijn. Anders werkt het niet of even traag als DPGS in vele toepassingen \cite{LBibRTK}. Bij RTK wordt de positie berkend door te werken met de fase van de draaggolf \cite{LBibRTK2}. De faser va nde draagglolg is de meer nauwkeurige versie va nde pseudo range. De draaggolf heeft normaal een constante frequentie. Maar de ontvangen draaggolf heeft een veranderende frequentie door het Doppler effect door de beweging van de satelliet en de ontvanger. Voor nauwkeurige positie bepaling worde nde metingen die door \'e\'en ontvanger gemaakt worden gecombineerd met de metingen van tweede ontvanger die simultaan gemaakt worden. In een RTK systeem gaan zowel het basis station en als het bewegende station bestaan uit een single -of dual GPS ontvanger, de geassocieerde antenne en aan data radio. Typisch gebruiken gebruikers identieke GPS ontvagetrs en data radios bij beide stations \cite{LBibRTK3}. 

\paragraph{Precise Point Positioning}
of PPP is een techniek die werkt met \'e\'en ontvanger en is zeer effici\"ent. \cite{LBibGNSS4}. PPP is momenteel gebaseerd op enkel GPS observaties. De nauwkeurigheid, beschikbaarheid en bertrouwbaarheid zijn sterk afhankelijk van het aantal zichtbare satellieten. Deze is vaak onvoldoende op bepaalde plaatsen. De nauwkeurigheid en de bertrouwbaarheid kan be\"invloed worden door slechte satelliet geometrie. Een mogelijke manier om de bschikbaarheid van satellieten te doen stijgen is om GPS en GLONASS resultaten te integregen. Daarnaar wordt momenteel veel onderzoek gedaan \cite{LBibPPP}. 

\subsection{GLONASS}
\label{LGLO}
GLONASS is \'e\'en van de langst gebruikte systemen binnen GNSS. Het is ontwikkeld door Rusland \cite{LBibGLONASS2}. GLONASS is een afkorting, het staat voor Globaluaya Naviagtsionnaya Sputnikovaya Sistema of in het Engels: GLObal NAvigation Satelite System  \cite{LBibBeiDou,LBibGNSS8}. GLONASS heeft veel gemeenschappelijk met GPS, met name de opbouw van de constellatie, de banen van de satellieten en signaal structuur. E\'en van de grootste verschillen tussen GLONASS en GPS is het referentie systeem waarin de co\"ordinaten geleverd worden \cite{LBibGNSS8}.Het GLONASS systeem is opgebouwd uit vier elementen:
\begin{itemize}
	\item Orbitale constellatie van GLONASS satellieten
	\item Controle segment op de grond
	\item Racket/ruimte complex
	\item Gebruikers
\end{itemize} \cite{LBibGLONASS2} Het systeem wordt continu gemoderniseerd \cite{LBibGNSS4}. GLONASS-M satellieten vormen een tweede genereatie van satellieten die gebruikt worden\cite{LBibGNSS}. De M staat voor Modfied (Aangepast). Deze satellieten werden in gebruik genomen in 2003 \cite{LBibPPP}.Deze tweede generatie heeft volgende voordelen \cite{LBibGLONASS,LBibPPP}:
\begin{itemize}
	\item Langere gegarandeerde levenstijd (zeven jaar in plaats van drie)
	\item Maakt gebruik van L2 signalen
	\item Stabielere klok
	\item Extra beschikbare navigatie data (zoals betere integritetiscontrole, informatie over absolute tijd beschikbaarheid van de satelliet nummer)
	\item inter-satellite radio link
	\item Betere zonnenpanelen postionering
	\item Lager niveau van onvoorspelde versnellingen
\end{itemize}
Door deze tweede generatie GLONASS-M satellieten zijn er nieuwe mogelijkheden voor satelliet navigatie. GLONASS is een betrouwbaar systeem, voornamelijk voor Real-Time Kinematic (RTK) mode in omgevingen met slechte zichtbaarheid \cite{LBibGLONASS}. Voor de werking van het navigatie satelliet systeem is het belangrijk dat alle processen die plaats vinden tijdens de werking gesynchroniseerd zijn. GLONASS-K satelieten vormen de derde generatie. De grootste veranderingen ten opzichte van GLONASS M zijn \cite{LBibGLONASS2}:
\begin{itemize}
	\item Het gebruik van een derde frequentie om betrouwbarheid en nauwkeurigheid voor gebruikersnavigatie te vergroten.
	\item De levensduur van de satelliet is vergroot tot 10 jaar
	\item Het gewicht van de satelliet is gehalveerd
\end{itemize}
Deze satellieten zijn vanaf 2008 in gebruik genomen \cite{LBibPPP}.

\subsubsection{Opbouw constellatie} 
De constellatie is opgebouwd uit 24 satelieten \cite{LBibGNSS4}. Deze satelliten zijn geplaatst in drie orbitale vlakken met een onderlinge afstand van 120 graden \cite{LBibGLONASS2,LBibGNSS6, LBibGNSS8}. Per vlak zijn er acht saellieten geplaatst. De periode per satelliet voor \'e\'en omwentelling rond de aarde is 11 uur en 15 minuten \cite{LBibGNSS6}.  Het systeem heeft zijn FOC bereikt in januari 1996 \cite{LBibGLONASS}. Het systeem is volledig gerevialiseerd en is volledig operationeel \cite{LBibGNSS4}. Een grafische weergave van de GLONASS constellatie is te zien in figuur \ref{imgGLONASS}.

\begin{figure}[hpb]
	\includegraphics[scale=0.5]{GLONASS.jpg}
	\caption{GLONASS constellatie \cite{LImgGLONASS}}
	\label{imgGLONASS}
\end{figure}

\subsubsection{Signalen}
Terwijl bij GPS de signalen mudolaties zijn van de draaggolven L1 en L2 voor alle satellieten, zijn bij GLONASS de draag frequenties afhankelijk van het uitzendende kanaal. Er zijn 12 kanalen voor 24 satellieten \cite{LBibGPS3}.  
 
\subsection{Galileo}
\label{LGal}
Galileo is het Europees systeem \cite{LBibGNSS3}\cite{LBibGNSS4}. Het doel van Galileo is het aanbieden van een flexibelere en nauwkeurige postionerings service \cite{LBibGNSS4}. Bij de ontwikkeling van het systeem is gespcificeert dat het stand-alone bruikbaar moet zijn, maar dat het eveneens moet kunnen samen werken met andere diensten zoals GPS. Het moet goede positie en timing diensten leveren \cite{LBibGalileo2}. De kostprijs wordt geschat tussen 2.2 en 2.9 bilioen Euro. De financiering komt van een gedeelde publieke en private vennootschap \cite{LBibGNSS8}. Galileo is ontworpen om Europa te voorzien van dezelfde GNSS capaciteit als dat van GPS \cite{LBibGNSS6}. De architectuur van Galileo specificeert een globaal integriteits concept. Dit wilt zeggen dat de nauwkeurigheid en de integriteit van de werking altijd wereldwijd bereikt moet worden en binnen de drempelwaarden moet blijven.  Het Galileo systeem voorziet verschillende gebruikers diensten. E\'en van deze diensten is de Safety of Life service, dit is een groot voordeel van Galileo  on vergelijking met GPS. In geval van een systeem fout moet de gebruiker binnen de zes seconden gewaarschuwd worden \cite{LBibGalileo}.

\subsubsection{Opbouw constellatie}
Tijdens het ontwikkelen van de opbouw van de constellatie is er geconcentreerd op twee opties. De eerste maaakt gebruik van MEO satellieten. De andere maakt gebruik van een mengeling tussen MEO en GEO satellieten. De nadruk ligt op het leveren van hoge kwaliteit diensten. Dit met name in Europa en de noordelijke regio's. Omdat de constellatie gebruikit wordt voor comerci\"ele en veiligheids toepassingen, is het ontwikkeld om zeer robuust te zijn en toch economisch verantwoord te zijn.  De constellatie die uiteindelijk gekozen is is deze met de MEO satellieten \cite{LBibGalileo2}. De volledige constellatie zal bestaan  uit 30 satelieten in drie orbitale vlakken. Het systeem is momenteel nog onder constructie \cite{LBibGNSS4}. De satellieten zijn geplaatst op drie orbitale vlakken. De periode van een tocht om de aarde is 14 uur en 21 minuten. Er zijn 10 satellieten per vlak \cite{LBibGNSS6}. De straal van de baan is 23223 km/ de satellieten zelf zijn van een gemiddelde grootte. Eens ze in hun baan zijn, wegen ze 650 kg en genereren een vermogen van 1500 Watt \cite{LBibGalileo2}.  Afbeelding \ref{imgGalileo} toont een grafische weergave van de constellatie.

\begin{figure}[hpb]
	\includegraphics[scale=1.75]{Galileo.jpg}
	\caption{Galileo constellatie \cite{LImgGalileo}}
	\label{imgGalileo}
\end{figure} 
 
\subsection{BeiDou}
\label{LBeD}
Bei Dou is het satilieten systeem binnen GNSS dat ontwikkeld is in China. Het systeem wordt vaak  aangeduid met de afkorting BDS \cite{LBibBeiDou} Het voorziet PNT (Postioning, Navia gtion and Timing) services in de Aziatisch-Pacifische regio. Momenteel is BDS  beshikbaar voor regionale diensten. Eveneens in 2020 zullen BDS signalen beschikbaar worden voor wereldwijde gebruikers. Ondertussen worden er steeds meer stations uitgebreid met BDS ontvangers voor hoog-precisie GNSS toepassingen \cite{LBibBeiDou}.

\subsubsection{Opbouw constellatie}
De volledige constellatie telt 35 satellieten. Het zal volledig zijn tegen het einde van 2020 \cite{LBibGNSS4}. De constellatie zal bestaan uit 5 GEo satellieten, 5 IGSO satellieten verdeelde over 2 banen en 4 MEO satellieten in een baan met 27 878km als straat \cite{LBibBeiDou2}. De opbouw van de constellatie wordt verduidelijkt in figuur \ref{imgBeiDou}.

\begin{figure}[hpb]
	\includegraphics[scale=0.5]{BeiDou.png}
	\caption{BeiDou constellatie \cite{LImgBeiDou}}
	\label{imgBeiDou}
\end{figure} 
\subsection{Besluit}
Eens de vier systemen volledig ingezet zijn, zijn er meer dan 100 satelieten beschikbaar voor nauwkeurige PNT toepassingen. Door het opkomen van de twee nieuwere systemen Galileo \ref{LGal} en BeiDou \ref{LBeD} en de het voortdurend moderniseren van GPS \ref{LGPS} en GLONASS \ref{LGLO} is de wereld van sateliet navigatie onderheving aan veranderingen \cite{LBibGNSS4}.Momenteel loopt er een Mutli-GNSS expirment (MGEX) dat data verzamelt van GPS?GLONASS, Galileo en BeiDou. De samenvoeging van multi-GNSS vergroot het aantal satellieten en bijgevolg wordt de geometrische observatie geoptimaliseerd \cite{LBibGNSS5}. De multiple-GNSS observaties met meerdere frequenties voorzien de onderzoekers van meerdere kansen om de Aardse ionosferische variaties en gedragingen te onderzoeken. GNSS observaties kunnen gebruikt worden om ionosferische traagheids correcties en gerelateerd atenschappelijk onderzoek \cite{LBibBeiDou}.

\section{EUREF Permanent Netwerk}
Het EUREF Permanent GNSS Network wordt ook wel EPN genoemd \cite{LBibEPN3,LBibEPN2,LBibEPN}. Het netwerk is gebaseerd op een relatie tussen site opperatoren van permanenten GNSS sites, die bereid zijn om hun data met het publiek te delen. Het EPN werkt nauw samen met het IGS \cite{LBibEPN3}. Het netwerk is opgebouwd uit 220 permanente GPS stations waarvan er 29 ook GLONASS satellieten volgend. Het primaire doel van EPN is het European Terrestrial Reference Systems (ETRS89) onderhouden, dit gebeurt door de leden op vrijwillige basis. \cite{LBibEPN}  \cite{LBibEPN2}. Een ander doel is het beschikbaar stellen van GNSS data en preciese co\"ordinaten van GNSS stations beschikbaar stellen aan het publiek. Nieuwe stations worden aan het netwerk toegevoegd zodra ze voldoen aan alle vereisten. Het EPN werkt met drie datacenters die gedefinieerd worden als regionale datacenters (RDC). Het Federal Agency for Cartography and Geodesy van Duitsland (BKG) en het Austrian Academy of Sciences (OLG)  datacenters, zijn verantwoordelijk voor de dagelijkse zaken. Alle EPN stations gaaan op vooraf gedefinieerde wegen hun data op regelmatige basis doorsturen naar het BKG en OLG \cite{LBibEPN2}. Het datacenter van Royal Observatory of Belgium (ROB) doet dienst als datacenter van het centrale bureau van het EPN. Dit datacenter is verantwoordelijk voor het hosten van alle historische EPN RINEX data met gecorrigeerde meta-data en 1 regionale broadcaster. \cite{LBibEPN2,LBibEPN3}. Het co\"ordineren van tijd analyse is eveneens een doel van het EPN. Dit om het EPN te versterken als geodetisch referentie netwerk. Van de nieuw ge\"installeerde antennes is 90 procent een multi-GNSS antenne. Hiervan is 75 procent ontworpen om GNSS en GLONASS te obeserveren, 25 procent is bovendien ook klaar om Galileo te ontvangen. Het netwerk is dus continu onder constructie. \cite{LBibEPN3}. Een station kan officieel erkend worden als een EUREF station als \cite{LBibGNSS8}:
\begin{itemize}
	\item Het station ge\"istalleerd is volgend de IGS standaard
	\item De log file van het station beschikbaar is op het EPN centraal bureau
	\item De data van het station beschikbaar zijn voor de EUREF gemeenschap
	\item De data van het station regelmatig geanalyseerd worden door een van de EUREF analyse centra. 
\end{itemize}

\section{Communicatie}
\label{LCom}
Er zijn verschillende manieren op GNSS data uit te wisselen. E\'en hiervan is RINEX, deze wordt besproken in sectie \ref{LRin}. Dit data formaat is niet beschikbaar voor real-time data transmissie. Om wel aan real-time transmissie te kunnen doen moeten er drie componenten aanwezig zijn. Transmissie protocol dat instaat voor de levering van de data over en netwerk en dat eveneens cotrole mechanismen voorziet. Data formaat is een overneekomst om de uitgezonden bit sequentie te kunnen vertalen naar zinvolle informatie. Als laatste date communicatie link dat instaat voor de manier waarop data van de ene naar de andere partij getransporteerd wordt. Er zijn twee standaard protocols om GNSS data over het internet te verdelen. Het eerste is RTCM, dit wordt verder besproken in sectie\ref{LRTC}. De tweede mogelijkheid in Rial-Time IGS (RTIGS) \cite{LBibRTCM}, dit wordt verder niet besproken in deze literatuurstudie. Dit omdat het tijdens het verder verloop van de literatuurstudie niet behandeld wordt.


\subsection{RINEX data formaat}
\label{LRin}
RINEX is een afkorting voor Reciever INdependant Exchange. Het is reeds in gebruik sinds het begin van GPS toepassingen. Het wisselt GPS data uit in een ASCII bestands formaat voor wetenschappelijke en geodatrische toepassingen. RINEX is een internationale standaard. Typisch worden uurlijkse of dagelijks data gearchiveer in RINX vestands formaat en wordt de data beschikbaar door een FTP server om te downloaden. Er zijn verschillende toepassingen die een post-processing methode gebruiken. RINEX is strikt bestand gebaseerd en is niet beschikbaar voor real-time data transmissie \cite{LBibRTCM}. 

\subsection{RTCM}
\label{LRTC}
RTCM staat voor Radio Technical Commission for Maritime Services \cite{LBibGLONASS}.De sterka vraag voor real-time DPGS heeft geleid tot de oprichting van het Special Comite 104 (afgekort tot SC104) \cite{LBibRTCM}. RTCM is het standaard formaat gebruikt voor RTK en DPGS correctie data. Het is momenteel het formaat dat het meest gebruikt wordt \cite{LBibRTK3}.
Momenteel zijn drie internationale standaarden voor RTCM \cite{LBibRTCM}:
\begin{itemize}
	\item RTCM SC 104
	\item RTCM SC104 2.3 (RTCM2.3)
	\item RTCM SC104 3.0 (RTCM3.0) 
\end{itemize}
RTCM 2.x heeft een onvoldoende data structuur, hierdoor wordt er een relatief hoge bandbreedte vereist, hierdoor was het niet volledig geschikt voor RTK operaties. Bij RTCM versie 2 was het formaat gebaseerd op de structuur van het GPS Navigatie beicht. Het GPS Naviagtie bericht wordt uitgezonden door een satelliet met een snelheid van 50Hz en \'e\'en woord bestaande uit 30 bits. RTCM versie 2 berichten bestaan uit twee of meer 30 bit woorden.De eerste twee woorden van elk bericht bevatten hoofding informatie zoals preamble, bericht type, station ID en dergelijke. Van de 30 bits kunnen 24 bits data dragen en 6 zijn gealloceerd voor parity check.De lengte van de berichten is gedefinieerd in de RTCM version 2 documenten. Er is een update gebeurd naar versie 3.0. Deze versie is ontwikkeld om RTK operaties te verbeteren en netwerk RTK te ondersteunen. Alle RTCM 3.0 berichten beginnen met een 8-bit vaste sequentie. Deze wordt gevolgd door 6 gereserveerde bits. Elke bericht lengte is variabe en is afhankelijk van het type bericht. Een overzicht van de structuur van een RTCM versie 3 bericht is terug te vinden in tabel \ref{TabRTCM}.

\begin{table}[hbp]
	\caption{RTCM versie 3 frame structuur}		
	\begin{tabular}{|c|c|c|c|c|}	
		\hline
		Preamble & Gereserveerd & Bericht lengte & Variabele lengte data bericht & CRC \\ \hline
		8 bits & 6 bits & 10 bits & Variabele lengte, geheel aantal bytes & 24 bits \\ \hline
	\end{tabular}
	\label{TabRTCM}
\end{table}

RTCM versie 3.0 is momenteel het meest populaire formaat. Maar RTCM versie 2 wordt nog steeds op veel plaatsen gebruikt, ondanks de vele voordelen van versie 3.0 \cite{LBibRTCM}.

\subsection{NTRIP}
\label{LNTR}
Networked Transport of RTCM via Internet Protocol of kortweg NTRIP \cite{LBibNTRIP,LBibNTRIP3}. NTTRIP is een protocol ontwikkeld in 2004 \cite{LBibNTRIP3}. Men maakt gebruik vant internet om de realtime GNSS data uit te wisselen en te verzamelen. NTRIP is een HTTP(Hypertext Transfer Protocol) stateless applicatie level protocol om GNSS te streamen over het internet \cite{LBibNTRIP}. De toepassing die men in gedachte had bij het ontwerpen van het protocol is het doorsturen van RTCM data, maar het protocol kan gebruikt worden om GNSS data in eender welk formaat door te sturen. De enige beperking is een maximum snelheid van 10 kb/s en een minimum van 100b/s \cite{LBibNTRIP3}. Deze techniek is ontwikkeld binnen het framwerk van EUREF. Het odeil is om real-time data uit te wisselen, maar ook om afgeleide producten te broadcasten, voornamelijk DGPS \cite{LBibNTRIP2}. De nodige brandbreedte hiervoor is niet groot in vergelijking met bijvoorbeeld Internet Radio\cite{LBibNTRIP}. Hierdoor wordt de NTRIP techniek een alternatief voor het verzenden van DGPS data tegenover andere  wereldwijde broadcasting technieken. Het internet is uitermate goed geschikt om data door te sturen tussen verschillende providers over een grote afstand \cite{LBibNTRIP2}. Het is gebaseerd op software die oorspronkelijk bedoelt was voor MP3 media speler formaten. Dit blijk wel aangepast te zijn voor GNSS stromen met data snelheden toosen 0.5 en 5 kbit/s. \cite{LBibGPS}. Een ander voordeel is dat tegenwoordig op veel plaatsen internet verbinding voorzien is \cite{LBibNTRIP}. Verder is er ook geen vermindering van de positie voorstelling door gebruik te maken van NTRIP. Een ander groot voordeel is dat datastromen van referentie stations simultaan beschikbaar worden via \'e\'en communicatie techniek. Men moet er wel rekening mee houden dat de nodige servers moeten verbonen worden aan het internet via intergeconnecteerde broadcasters met een voldoende grote bandbreedte.  De generatie van DPGS correctie data wordt meestal direct op de GPS ontvanger gedaa, maar hij kan eveneens bepaald worden van observaties die verkregen worden dooe verschillende referentie stations in het netwerk. De datastroom wordt dan door gegeven aan een server die de stroom vervolgens beschikbaar stelt over her internet via een geschikt protocol \cite{LBibNTRIP2}. Het gevolgde pad van de data wordt weergegeven in figuur \ref{imgNTRIP2}.
\begin{figure}[hpb]
	\includegraphics[scale=0.4]{NTRIP2.png}
	\caption{RTCM data stroom op het internet \cite{LBibNTRIP2}}
	\label{imgNTRIP2}
\end{figure} 
De afstand tussen het referentie staion en de client met het verbonden station is opgedeeld in twee delen. Het groote deel van de afstand bestaat uit een bedrade internet verbinding. Het overige deel wordt overbrugd door mobiele telefonie tehnologie \cite{LBibNTRIP2}.

\subsubsection{opbouw van NTRIP software}
\label{LONS}
NTRIP wordt ge\"implemeneteerd in vier grote componenten. Nameijk NTRIP Sources, Ntrip clients, NtripServers en NtripCasters. De NTRIP Sources stellen de bronnen van de GNSS data voor. Deze worde ngevoed aan het systeem/ Meestal is dit een GNSS ontvanger die obdervaties en voorziet of die DGPS correctie data genereert \cite{LBibNTRIP3}. De NtripCaster is het HTTP server programma terwijl NTRIPClients en NTRIPServers reageren zoals HTTP clients\cite{LBibNTRIP}. NTRIPServers gaat datastromen vervoeren. NTRIPCasters gaan de administratie tussen clients en servers afhandelen \cite{LBibGPS}. NTRIPCasters is een stream-spliteser en broadcaster component, momenteel zijn er acht NTRIPCasters in Europa \cite{LBibNTRIP}. De casters gaan zoals vaak gebeurt bij internet radio implementaties de binnenkomende datastromen dupliceren, zodat hij door meerdere gebuikers simultaan ontvangen kan worden \cite{LBibNTRIP2}.NTRIP omvat de voorziening van metadata door een Soaurce Table die onderhouden wordt door de NTRIP Caster \cite{LBibNTRIP3}.  NTRIP Clients gaan data ontvangen van de gewenste bronnen via de NTRIPCaster. NTRIPServers gaan de data van \'e\'en of meerdere bronnen verzenden in NTRIP formaat. Als laatste hebben we ook nog NTRPSorurces, deze genereren DPGS datastreams op een specifieke locatie \cite{LBibNTRIP}. In figuur \ref{imgNTRIP} staat een overzicht van de opbouw van het NTRIP streaming systeem. NTRIP is gebaseerd op een subset van het  vaak gebruikte HTTP an is bijgevolg dus gebaseerd op TCP. Daaruit volgt dat het streamen gebeurt vanuit een enkele IP Poort. Meestal is dit poort 80 of 2101 \cite{LBibNTRIP3}. 

\begin{figure}[hpb]
	\includegraphics[scale=0.65]{NTRIP.jpg}
	\caption{NTRIP streaming systeem \cite{LBibNTRIP}}
	\label{imgNTRIP}
\end{figure} 
 



